\section*{Notation}

\noindent\textbf{Conventions.}  
Time and entropy advance in discrete, \emph{dimensionless ticks}.   
A single primitive cell unit is created per tick, giving
\[
  \Delta S = \Delta \tau.
\]
Physical mapping is applied only when necessary:
\[
  S_{\mathrm{phys}} = k_B\,S, \qquad
  \tau_s = t_P\,\tau.
\]
Fundamental constants: speed of light \(c\), reduced Planck constant \(\hbar\), and Boltzmann constant \(k_B\).

\paragraph{Notation guard.}  
\(\mathcal Q\) denotes \emph{channel throughput capacity} (J·m), distinct from heat \(Q\) and electric charge \(q\).  
Electrostatic potential is written \(\varphi\) (to avoid collision with the phase symbol \(\phi\)).

\smallskip
\noindent\textit{Native--SI correspondence.}  
One hop per tick implies \(\mathrm{d}a/\mathrm{d}\tau = 1\).  
Choosing a length per hop \(\ell_0\) and a time per tick \(t_0\) yields
\[
  c = \frac{\ell_0}{t_0}.
\]

\bigskip
\noindent\textbf{Core quantities.}
\begin{description}[leftmargin=2.4em,labelsep=0.8em]
  \item[\(k\)] Tick index (integer). The scale factor \(a(k)\) is defined with respect to ticks; baseline \(a(k)\propto k^{1/3}\).
  \item[\(\Delta S,\,\Delta\tau\)] Primitive-cell and proper-time tick \emph{counts} (dimensionless) satisfying \(\Delta S=\Delta\tau\).
  \item[\(\alpha\)] Fine-structure constant,
        \[
          \alpha = \frac{e^2}{4\pi\epsilon_0\hbar c}
                 = \frac{Z_0}{2R_K}.
        \]
  \item[\(Q_{\max}\)] Per-tick action-flux scale, \(Q_{\max} := \hbar c\) (J·m).
  \item[\(e\)] Elementary charge (C).
  \item[\(\epsilon_0\)] Vacuum permittivity (F·m\(^{-1}\)).
  \item[\(\mu_0\)] Vacuum permeability (H·m\(^{-1}\)).
  \item[\(Z_0\)] Vacuum impedance, \(Z_0 = \mu_0 c \approx 376.730~\Omega.\)
  \item[\(R_K\)] von Klitzing constant, \(R_K = h / e^2 \approx 25{,}812.807~\Omega.\)
  \item[\(\delta\phi\)] Minimal gauge-phase increment, \(\alpha = (\delta\phi)^2 / 4\pi.\)
\end{description}

\noindent\textbf{Graph and geometric parameters.}
\begin{description}[leftmargin=2.4em,labelsep=0.8em]
  \item[\(d_G(x,y)\)] Graph (hop) distance between vertices \(x\) and \(y\).
  \item[\(\ell_N\)] Typical interior hop length for a sample of size \(N\).
  \item[\(\varepsilon_N,\,\delta_N\)] Mesh non-uniformity and directional-bias parameters (\(\to 0\) as \(N\to\infty\)).
  \item[\(\eta_N\)] Combined small parameter: \(\eta_N := C_1\varepsilon_N + C_2\delta_N.\)
  \item[\(\mathrm{distortion}(N)\)] Bilipschitz distortion of the graph metric relative to the ambient metric:
  \[
    \mathrm{distortion}(N)
      = \sup_{x,y}
        \max\!\left\{
          \frac{d_G(x,y)}{\|x-y\|},\,
          \frac{\|x-y\|}{d_G(x,y)}
        \right\}-1.
  \]
\end{description}

\noindent\textbf{Electromagnetics-specific quantities.}
\begin{description}[leftmargin=2.4em,labelsep=0.8em]
  \item[\(\mathbf{E},\,\mathbf{B}\)] Electric and magnetic field vectors; \(F_{\mu\nu}\) denotes the field tensor where appropriate.
  \item[\(\nu,\,\omega,\,k,\,\lambda\)] Frequency, angular frequency, wavenumber, and wavelength, related by \(k = 2\pi/\lambda = \omega/c.\)
  \item[\(\phi,\,\delta\phi\)] Wave phase and residual phase shift, \(\delta\phi = \phi_{\text{meas}} - kL.\)
  \item[\(n(\lambda),\,v_g\)] Refractive index and group velocity.
  \item[\(\alpha\)] Fine-structure constant (dimensionless electromagnetic coupling).
  \item[\(Q_{\max}\)] Per-tick action flux scale \(\hbar c\).
  \item[\(e,\,\epsilon_0,\,\mu_0,\,Z_0,\,R_K\)] Standard electromagnetic constants.
  \item[\(\mathcal Q\)] Channel throughput capacity (J·m) in tick-space representation.
\end{description}

\bigskip
\noindent\textbf{Quick symbol summary.}
\begin{description}[leftmargin=2.4em,labelsep=0.8em]
  \item[\(\ell_P\)] Planck-cell edge or hop length.
  \item[\(t_P\)] Planck tick (one global update interval).
  \item[\(c=\ell_P/t_P\)] Causal speed cap (one hop per tick).
  \item[\(k\)] Discrete tick index (global ledger step).
  \item[\(R(k)\)] Front radius after \(k\) ticks.
  \item[\(N(k)\)] Number of active/born cells after \(k\) ticks.
  \item[\(L\)] Continuum baseline distance.
  \item[\(d_G\)] Graph (hop) distance.
  \item[\(T(L)\)] Transit time over baseline \(L\).
  \item[\(v_g\)] Group or front speed.
  \item[\(E,\,\hbar\)] Energy and reduced Planck constant.
  \item[\(\Delta\phi\)] Accumulated phase along the path.
  \item[\(\lambda\)] Wavelength of the probe signal.
\end{description}
